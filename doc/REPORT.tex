\documentclass[a4paper,12pt]{book}%,twocolumn
%\documentclass[a4paper,11pt]{article}
\input{settings/packages}
\input{settings/page}
\input{settings/macros}
\usepackage{algpseudocode}
\usepackage{float}
\begin{document}

\begin{titlepage}
\center % Center everything on the page

%-------------------------------------------------------------------------------------
%	HEADING SECTIONS
%------------------------------------------------------------------------------------
\textbf{\Large L.E. Robotics (Pvt.) Ltd.}\\[0.5cm]
\textbf{\large No. 100/4, Divulapitiya Road, Minuwangoada,	Sri Lanka}\\[3cm]

\includegraphics[width=0.3\textwidth]{figures/logoler}\\[3cm]

	
%-------------------------------------------------------------------------------------
%	TITLE SECTION
%------------------------------------------------------------------------------------
\textbf{\Huge Machine Vision based\\ Real Time Trajectory Generation\\ (MVbR2TG)}\\[6cm]
%\textbf{\Large A comparison}\\[7cm]


%----------------------------------------------------------------------------------------
%	MEMBERS SECTION
%----------------------------------------------------------------------------------------




\begin{tabular}[!h]{ l l}
\textbf{\large Prepared by} & {\large Thalagala B.P.}\\
\textbf{\large Last Modified on}&  {\large \today}
\end{tabular}

%\textbf{\large Prepared by}\\[0.5cm]
%{\large Thalagala B.P.}\\[1cm]
%
%%----------------------------------------------------------------------------------------
%%	DATE SECTION
%%----------------------------------------------------------------------------------------
%\textbf{\large Last Modified on}\\[0.5cm]
%\textbf{\Large \today} % Date, change the \today to a set date if you want to be precise

%----------------------------------------------------------------------------------------


\end{titlepage}
\tableofcontents




\chapter{Feasibility Study}
\section{Introduction}
Most of the robotic arms used in industrial environments operate in a pre-programmed cycle. When it comes to the way a human does the same task is much different as the path planning for picking an object may change from cycle to cycle because of the perception obtained through human vision.\\

Machine vision is the technology and methods incorporated to mimic the human vision in order to gain the insights about the operating environment of the robotics system. However, when it comes to the real time object detection using machine vision, there is an inevitable trade-off between the accuracy and the speed of the operation. This depends entirely on the used machine vision algorithms and the computational power of the available hardware.

\section{Challenges Encountered when Incorporating  a Vision Unit to a Robotic System}

If the robotic system/ arm in interest is not controlled through a dedicated industrial PC with adequate computational resources, the amount of resources that can be allocated to the vision unit becomes limited.  This will eventually result in great delays (which is not desirable when it comes to real time operations) to produce the required outputs by processing the acquired images through the associated camera.\\

Due to this limitation in computational resources it is not practical to use Deep Learning based algorithms in time critical industrial applications. Because, by the time the decisions are made the environment may have already changed and the made decisions may no longer valid.
\pagebreak
\section{A Traditional Machine Vision Approach as the Solution}

To overcome the challenges mentioned above traditional machine vision algorithms can be incorporated inside the vision unit. Unlike the deep learning based algorithms, traditional machine vision algorithms  are less computationally expensive and no initial data is require to train the algorithm. As they are build upon strong mathematical foundation they work well in almost all the situations when the parameters are tuned properly.

\section{High Level Representation of the Algorithm}

For real time trajectory generation we first need to track the objects on the conveyor belt. A typical object tracking framework generally consists of three major modules which can be identified as, 1. Object Detection, 2. Object Modeling and 3. Object Tracking. The following traditional computer vision methods will be used to implement those three modules.\\

\begin{table}[!h]
	\centering
	\begin{tabular}[!h]{|l | l|}
		\hline
		\textbf{Module} & \textbf{Method}\\
		\hline
		Object Detection &  Connected Component Analysis and Background Subtraction\\
		Object Modeling & Contour Analysis and Representation\\
		Object Tracking & Template Matching followed by Object moments interpretation\\ 
		\hline\hline
	\end{tabular}
\caption{Modules of a Typical Object Tracking Framework}
\end{table}
	
%Pseudo code representation of the functioning of above three modules cane be expressed as follows.
%
%
%\begin{verbatim}
%1. Capture an image of a sample object placed on the conveyor belt
%2. Create the `template image' using that image
%3. While True
%4. 	Capture a frame
%5. 	Convert the frame to a grayscale image
%6. 	Thresholding the image for segmentation
%7. 	Connected compponent Analysis
%8. 	Background Subtraction
%9. 	Contour Analysis to identify the object Contorurs
%10. Template matching 
%\end{verbatim}

Through the above framework we can identify and track the centroids of the objects on the conveyor belt and that information cam be handed over to the trajectory generation algorithm to generate the required path in the real time  to pick the object. Following subsections highlights the contribution from the each module tabled above where a conveyor belt which carries hexagonal nuts is used as the example.

\begin{figure}[H]
	\centering
	\includegraphics[scale=0.17]{figures/belt}
	\caption{Single frame of the video captured from the camera}
\end{figure}


\subsection{Object Detection: Connected Component Analysis (CCA) and Background Subtraction}

As the first step of the object detection and tracking, the foreground (objects) must be separated from the background. This segmentation is done at this stage  using \textit{connected component analysis} (CCA). Through the CCA different connected objects found in the image can be assigned different labels. Those label information can then be used to subtract the background from the  image. Prior to CCA, thresholding followed by a suitable morphological transformation must be done. For this example \textit{Otsu's thresholding} and \textit{morphological closing} are used.

\begin{figure}[H]
	\centering
	\subfigure[Labeling of connected objects using CCA]
	{ \includegraphics[scale=0.65]{figures/cca}
	}\hspace{5mm}
	\subfigure[Subtracting the background]
	{ \includegraphics[scale=0.65]{figures/backsub}
	}
\caption{Connected Component Analysis (CCA) and Background Subtraction}
\end{figure}

\subsection{Object Modeling: Contour Analysis and Representation}

Background subtracted image is used in this stage to find contours. These contours can be then used as a representation of the objects on the conveyor belt.

\begin{figure}[H]
	\centering
	\includegraphics[scale=0.65]{figures/contours}
	\caption{Finding contours in the image}
\end{figure}

\subsection{Object Tracking: Template Matching followed by Object moments interpretation}

In order to identify different objects a procedure called as \textit{template matching} is carried out. For this template matching, reference images known as template images should be created using a properly captured images of the objects that are intended to convey using the conveyor belt. Contours of these templates can be compared with the object contours found in the previous stage to identify similar objects for further processing. 

\begin{figure}[H]
	\centering
	\includegraphics[scale=0.45]{figures/template}
	\caption{Sample template image for a conveyor belt carrying hexagonal nuts}
\end{figure}

\subsection{Related functions in OpenCV computer vision library}
\begin{enumerate}[1.]
	\item {\tt cv.threshold}
	\item {\tt cv.morphologyEx}
	\item {\tt cv.connectedComponentsWithStats}
	\item {\tt cv.findContours}
	\item {\tt cv.drawContours}
	\item {\tt cv.matchShapes}
	\item {\tt cv.moments}
	
	
\end{enumerate}

\chapter{Underlying Concepts of the Algorithms}

\section{Object Detection}

\section{Object Modeling}

\subsection{Paper: Topological Structural Analysis of Digitized Binary	Images by 	Border Following}

\textbf{\textit{Authors of the Paper : SATOSHI SUZUKI, KEIICHI ABE}}\\

Implementations found on GitHub:\\

{\footnotesize \url{https://github.com/RaphaelJ/friday/blob/master/src/Vision/Image/Contour.hs}

\url{https://github.com/lagadic/visp/blob/master/modules/imgproc/src/vpContours.cpp} 

\url{https://github.com/lagadic/visp/blob/master/modules/imgproc/include/visp3/imgproc/vpContours.h}

\url{https://github.com/JuliaImages/juliaimages.github.io/blob/source/docs/examples/contours/contour_detection.jl}}

\subsubsection{Abstract}

``Two border following algorithms are proposed for the topological analysis of digitized binary images. The first one determines the surroundness relations among the borders of a binary image. Since the outer borders and the hole borders have a one-to-one correspondence to the connected components of l-pixels and to the holes, respectively, the proposed algorithm yields a representation of a binary image, from which one can extract some sort of features without reconstructing the image. The second algorithm, which is a modified version of the first, follows only the outermost borders (i.e., the outer borders which are not surrounded by holes). These algorithms can be effectively used in component counting, shrinking, and topological structural analysis of binary images, when a sequential digital computer is used. o 1985 Academic press, IC.''\\

\subsubsection{INTRODUCTION}
In this paper, they first present an algorithm which can extract the topological structure of a given binary image. This algorithm is an extended version of the border following algorithm, which discriminates between outer borders and hole borders.\\


The extensions are:\\
(1) to put a unique mark on each border rather than to adopt the same marking procedure for every border (border labeling); and\\
(2) to add a procedure for obtaining the parent border of the currently followed border\\

Next they show a modified version of the first algorithm which follows only the outermost borders of a binary image.


\subsubsection{BASIC CONCEPT AND NOTATIONS}

In this paper only digital binary pictures sampled at points of rectangular grids are considered.\\

The pixel located in the $i$ th row and the $j$ th column is represented by the row number and the column number $(i, j)$. They adopt the following coordinate system:\\

The row number $i$ increases from top to bottom; the column number $j$ from left to right. A picture having the density value
$f_{ij}$ at a pixel $(i,j)$ is denoted by $F= \left\{ f_{ij} \right\}$.\\

A l-component and a 0-component are the connected components of l-pixels and of 0-pixels, respectively. If a 0-component S contains the frame of the picture, they call S the background; otherwise, a hole.\\

It is well known that in order to avoid a topological contradiction 0-pixels must be regarded as 8- (4-) connected if l-pixels are dealt with as 4- (8-) connected. We will say ``\textbf{\textit{in the 4- (8-) connected case}}'' when we deal with l-pixels as 4- (8-) connected
and O-pixels as 8- (4-) connected.\\



\textbf{\textit{4-connected pixels}} are neighbors to every pixel that touches one of their edges. These pixels are connected horizontally and vertically. In terms of pixel coordinates, every pixel that has the coordinates $(x \pm 1, y )$ or $(x, y \pm 1)$ is connected to the pixel at $(x,y)$.\\

\textbf{\textit{8-connected pixels}} are neighbors to every pixel that touches one of their edges or corners. These pixels are connected horizontally, vertically, and diagonally. In addition to 4-connected pixels, each pixel with coordinates $(x \pm 1, y \pm 1 )$ is connected to the pixel at $(x,y)$.\\

\subsubsection{THE BORDER FOLLOWING ALGORITHM FOR TOPOLOGICAL ANALYSIS}
We present a border following algorithm for topological structural analysis. This extracts the surroundness relation among the borders of a binary picture. First we give an informal explanation of the algorithm.

\paragraph{Algorithm 1}

\begin{enumerate}
	\item Scan an input binary picture with a TV raster (In raster scanning, the beam sweeps horizontally left-to-right at a steady rate, then blanks and rapidly moves back to the left, where it turns back on and sweeps out the next line. During this time, the vertical position is also steadily increasing (downward), but much more slowly. There is one vertical sweep per image frame, but one horizontal sweep per line of resolution. - From Wikipedia, the free encyclopedia)
	
	\item Interrupt the raster scan when a pixel $(i, j)$ is found which satisfies the condition for the border following starting point of either an outer border (Fig. 2a) or a hole border (Fig. 2b).
	
	\item If the pixel $(i, j)$ satisfies both of the above conditions, $(i, j)$ must be regarded as the starting point of the outer border. Assign a uniquely identifiable number to the 	newly found border. Let us call it the \textit{sequential number of the border} and denote it by \textbf{NBD}.
	
	\item Determine the parent border of the newly found border as follows. During the	raster scan we also keep the sequential number \textbf{LNBD} of the (outer or hole) \textit{border	encountered most recently}. This memorized border should be either the parent	border of the newly found border or a border which shares the common parent with	the newly found border. Therefore, we can decide the sequential number of the	parent border of the newly found border with the types of the two borders according	to Table 1.
	
	\item Follow the found border from the starting point, marking the pixels on the border. 	The border following scheme is the classical one [l-3]; the distinguished feature of	our algorithm is the marking policy.
	
	\begin{enumerate}[(a)]
		\item If the current following border is between the O-component which contains	the pixel ( p, q + 1) and the l-component which contains the pixel ( p, q), change the	value of the pixel (p, q) to - NBD.
		
		\item Otherwise, set the value of the pixel (p, q) to NBD unless (p, q) is on an	already followed border
	\end{enumerate}
	
	\item The conditions (a) and (b) prohibit the pixel ( p, q) from being the border following	starting points of the already followed hole border and outer border, respectively.The positive value NBD and the negative value, -NBD, correspond to the labels
``l'' and ``r'' of the border following algorithm in [l], respectively.
	\item After following and marking the entire border, resume the raster scan. When the	scan reaches the lower right comer of the picture, the algorithm stops.
\end{enumerate}


\paragraph{THE FORMAL DESCRIPTION OF ALGORITHM 1}


\section{Object Tracking}



\bibliographystyle{plain}
\bibliography{refer}

%---------------------------------------------------------------------------
\end{document}
