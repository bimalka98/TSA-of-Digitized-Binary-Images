\documentclass[a4paper,11pt]{article}%,twocolumn
%\documentclass[a4paper,11pt]{article}
\input{settings/packages}
\input{settings/page}
\input{settings/macros}


\begin{document}
	
\begin{titlepage}
\center % Center everything on the page

%-------------------------------------------------------------------------------------
%	HEADING SECTIONS
%------------------------------------------------------------------------------------
\textbf{\Large L.E. Robotics (Pvt.) Ltd.}\\[0.5cm]
\textbf{\large No. 100/4, Divulapitiya Road, Minuwangoada,	Sri Lanka}\\[3cm]

\includegraphics[width=0.3\textwidth]{figures/logoler}\\[3cm]

	
%-------------------------------------------------------------------------------------
%	TITLE SECTION
%------------------------------------------------------------------------------------
\textbf{\Huge Machine Vision based\\ Real Time Trajectory Generation\\ (MVbR2TG)}\\[6cm]
%\textbf{\Large A comparison}\\[7cm]


%----------------------------------------------------------------------------------------
%	MEMBERS SECTION
%----------------------------------------------------------------------------------------




\begin{tabular}[!h]{ l l}
\textbf{\large Prepared by} & {\large Thalagala B.P.}\\
\textbf{\large Last Modified on}&  {\large \today}
\end{tabular}

%\textbf{\large Prepared by}\\[0.5cm]
%{\large Thalagala B.P.}\\[1cm]
%
%%----------------------------------------------------------------------------------------
%%	DATE SECTION
%%----------------------------------------------------------------------------------------
%\textbf{\large Last Modified on}\\[0.5cm]
%\textbf{\Large \today} % Date, change the \today to a set date if you want to be precise

%----------------------------------------------------------------------------------------


\end{titlepage}
\tableofcontents
\pagebreak




%\begin{figure}[!h]
%	\centering
%	\includegraphics[scale=0.45]{figures/macllc}
%	\caption{10BASE-T relationship to the ISO/IEC Open Systems Interconnection (OSI) reference model and the IEEE 802.3 CSMA/CD LAN model\cite{main}}
%\end{figure}

\section{Introduction}
Most of the robotic arms used in industrial environments operate in a pre-programmed cycle. When it comes to the way a human does the same task is much different as the path planning for picking an object may change from cycle to cycle because of the perception obtained through human vision.\\ 

Machine vision is the technology and methods incorporated to mimic the human vision in order to gain the insights about the operating environment of the robotics system. However, when it comes to the real time, object detection using machine vision, there is an inevitable trade-off between the accuracy and the speed of the operation. This depends entirely on the used machine vision algorithms and the computational power of the used hardware.

\section{Feasibility Study}
%\bibliographystyle{plain}
%\bibliography{refer}

%---------------------------------------------------------------------------
\end{document}